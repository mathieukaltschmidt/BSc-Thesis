\documentclass[border = 0.5cm]{standalone}
\input{feynman_settings}

\usetikzlibrary{patterns, decorations.markings, snakes, calc, decorations.pathreplacing, decorations}

\usepackage{relsize} %larger math symbols

\begin{document}

%Setup for the wiggly circle: The idea is to add a sine function to the usual circle 
\tikzset{/pgf/decoration/.cd,
    number of sines/.initial=10,
    angle step/.initial=20,
}
\newdimen\tmpdimen
\pgfdeclaredecoration{complete sines}{initial}
{
    \state{initial}[
        width=+0pt,
        next state=move,
        persistent precomputation={
            \pgfmathparse{\pgfkeysvalueof{/pgf/decoration/angle step}}%
            \let\anglestep=\pgfmathresult%
            \let\currentangle=\pgfmathresult%
            \pgfmathsetlengthmacro{\pointsperanglestep}%
                {(\pgfdecoratedremainingdistance/\pgfkeysvalueof{/pgf/decoration/number of sines})/360*\anglestep}%
        }] {}
    \state{move}[width=+\pointsperanglestep, next state=draw]{
        \pgfpathmoveto{\pgfpointorigin}
    }
    \state{draw}[width=+\pointsperanglestep, switch if less than=1.25*\pointsperanglestep to final, % <- bit of a hack
        persistent postcomputation={
        \pgfmathparse{mod(\currentangle+\anglestep, 360)}%
        \let\currentangle=\pgfmathresult%
    }]{%
        \pgfmathsin{+\currentangle}%
        \tmpdimen=\pgfdecorationsegmentamplitude%
        \tmpdimen=\pgfmathresult\tmpdimen%
        \divide\tmpdimen by2\relax%
        \pgfpathlineto{\pgfqpoint{0pt}{\tmpdimen}}%
    }
    \state{final}{
        \ifdim\pgfdecoratedremainingdistance>0pt\relax
            \pgfpathlineto{\pgfpointdecoratedpathlast}
        \fi
   }
}


%actual code

\raisebox{4ex}{\hspace{-0.7cm}$\partial_t\Gamma_k[\phi] = \mathlarger{\frac{1}{2}}$
 }

\begin{tikzpicture}[cross/.style={path picture={ 
\draw[black]
(path picture bounding box.south east) -- (path picture bounding box.north west) (path picture bounding box.south west) -- (path picture bounding box.north east);
}}]
\draw[thick] circle(0.7);
\node [draw, fill = white, circle, cross, minimum width= 0.1 cm] at (0,0.7){}; 
\node [draw, fill = BScGray!30, circle] at (0,-0.7){}; 
\end{tikzpicture}


\end{document}
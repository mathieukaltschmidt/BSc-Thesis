\chapter{Mathematical Appendix}
In this part of the appendix we want to present some of the mathematical tools we used during the calculations presented in the scope of this thesis in a more formal manner.

\section{York Decomposition}
\blindtext

\section{Heat-Kernel Techniques}\label{sec:heat-kernel}
We use heat-kernel techniques to evaluate the r.h.s. of the flow equation (\ref{eqn:Wetterich}), where we need to compute traces over functions depending on the Laplacian on a curved background. In general, the method can be understood as a curvature expansion about a flat background. \\
The formula to compute such traces is given by
\begin{align}
	\operatorname{Tr} f(\Delta)= N \  \int\kern-1.3em\sum_{\ell} \rho(\ell) f(\lambda(\ell)),
	\label{eqn:heat-kernel}
\end{align}
with some normalization $N$, the spectral values $\lambda(\ell)$ and their corresponding multiplicities $\rho(\ell)$. \\
On flat backgrounds, the computation of (\ref{eqn:heat-kernel}) is simply a standard momentum integral, whereas on curved backgrounds, consider for example a four-sphere with constant background curvature $r = \frac{\bar{\mathcal{R}}}{k^2} > 0$, the spectrum of the Laplacian is discrete and we need to sum over all spectral values in (\ref{eqn:heat-kernel}). \\
For our example of the four-sphere, we have
\begin{align}
	\lambda(\ell) = \frac{\ell(3+\ell)}{12}r \qquad \text{and} \qquad \rho(\ell) = \frac{(2\ell + 3)(l+2)!}{6\ell!}.
\end{align}
and the normalization is given by the inverse of the four-sphere-volume $N= V_{S^4}^{-1} = \frac{k^4r^2}{384\pi^2}$. This leads us to the formula for our computation of the r.h.s. of the flow equation on a background with constant positive curvature
\begin{align}
\operatorname{Tr} f(\Delta)=\frac{k^{4} r^{2}}{384 \pi^{2}} \sum_{\ell=0}^{\infty} \frac{(2 \ell+3)(\ell+2) !}{6 \ell !} f\left(\frac{\ell(3+\ell)}{12} r\right)	.
\end{align}
This is called the spectral sum. For large curvatures $r$ the convergence of the series is rather fast, whereas in the limit $r\rightarrow 0$ one finds exponentially slow convergence.\\
The master equation for heat kernel computations reads
\begin{align}
	\operatorname{Tr} f(\Delta)=\frac{1}{(4 \pi)^{\frac{d}{2}}}\left[B_{0}(\Delta) Q_{2}[f(\Delta)]+B_{2}(\Delta) Q_{1}[f(\Delta)]\right]+O\left(\bar{\mathcal{R}}^{2}\right)
\end{align}
with the heat-kernel coefficients 
\begin{align}
	B_{n}(\bar{\Delta})=\int \mathrm{d}^{d} x \sqrt{\operatorname{det}\bar{g}} \  \operatorname{Tr} b_{n}(\bar{\Delta})
\end{align}
and 
\begin{align}
	Q_{n}[f(x)]=\frac{1}{\Gamma(n)} \int \mathrm{d} x x^{n-1} f(x).
\end{align}
For our computation on $S^4$, the traces over the coefficients $b_n(\bar{\Delta})$ are presented in the following.
\begin{table}[H]
	\centering
	\setlength{\tabcolsep}{5mm}
	\setlength\extrarowheight{2mm}
	\begin{tabular}{c | c c c}
	   & TT & TV & S\\ \hline
	   $\operatorname{Tr} b_{0}$ & 5 &  3 & 1\\
	  $\operatorname{Tr} b_{2}$ & $-\frac{5}{6}\mathcal{R}$ & $\frac{1}{4}\mathcal{R}$& $\frac{1}{6}\mathcal{R}$\\
	\end{tabular}
	\caption{Heat-kernel coefficients for transverse-traceless tensors (TT), transverse vectors (TV) and scalars (S) for computations on the four-sphere $S^4$.}
\end{table}

The basic idea of the proof of equation (\ref{eqn:heat-kernel}) is based on the Laplace transform
\begin{align}
	f(\Delta) = \int_0^{\infty} \dd s \operatorname{e}^{-s\Delta}\tilde{f}(s).
\end{align}
%TODO: Finish this proof. Add entire calculation.



\chapter{Mathematical Background}\label{chap:AppA}
In this part of the appendix we want to discuss some of the mathematical tools we used during the calculations presented in the scope of this thesis in a more formal manner. 
The part on the York decomposition is mainly inspired by \cite{Percacci2017}, whereas the conventions for the heat-kernel computations are taken from \cite{PawlowskiNPgaugeLecture} and extended for the matter part, using the conventions from \cite{CodelloPercacciRahmede2008}.

\section{York Decomposition}
In the discussion of gauge theories, it is often very useful to decompose the gauge field $A_{\mu}$ into transversal and longitudinal parts:
\begin{align}
	A_{\mu} = A_{\mu}^{\mathrm{T}} + \nabla_{\mu}\phi.
\end{align}
The transversal part is characterized by the fact, that $\nabla^{\mu}A_{\mu}^{\mathrm{T}} = 0$. Using this decomposition, we are able to separate the pure gauge spin-$0$ degrees of freedom from the physical ones, contained in the spin-$1$ part $A_{\mu}^{\mathrm{T}}$.\\
Assuming vanishing boundary terms, integration by parts allows us to change the integration variables in the functional integral, i.\,e.
\begin{align}
	\int_x \sqrt{g} \ A_{\mu}A^{\mu} = \int_x \sqrt{g} \ A_{\mu}^{\mathrm{T}}A^{\mathrm{T}, \mu} + \int_x \sqrt{g} \ \phi\left(-\nabla^2\right)\phi.
\end{align}
Note, that we have to take care of the Jacobian $J$ of this variable transformation:
\begin{align}
	\left(\dd A_{\mu}\right) \longrightarrow J\left(\dd A_{\mu}^{\mathrm{T}}\right)\left(\dd\phi\right).
\end{align}
To be able to determine the Jacobian for our transformation, the integration measure needs to be normalized. A quite convenient choice is to evaluate the Gaussian integral over the different fields $\psi$ and set the result to one:
\begin{align}\label{eqn:york_measure}
	\int\left(\dd\psi\right) \exp\left\{-\int\dd x \ \sqrt{g} \ \psi^2 \right\} = 1,
\end{align} 
where we are assuming an Euclidean signature and a curved background metric. With this condition we find:
\begin{align}
	1&=J \int\left(\dd A_{\mu}^{\mathrm{T}}\right) \operatorname{e}^{-\int \dd x \sqrt{g} \ A_{\mu}^{\mathrm{T}} A^{\mathrm{T}, \mu}} 
	\int(\dd\phi) \operatorname{e}^{-\int \dd x \sqrt{g} \  \phi\left(-\nabla^{2}\right) \phi} = J\left(\operatorname{det}_{\phi}^{\prime}\left(-\nabla^{2}\right)\right)^{-1/2}.
\end{align}
This allows us to determine the Jacobian $J$ as follows:
\begin{align}
	J = \left(\operatorname{det}_{\phi}^{\prime}\left(-\nabla^{2}\right)\right)^{1/2}.
\end{align}
The prime denotes the fact, that the zero mode has to be removed, when computing the determinant to obtain a consistent result. Physically this is in accordance with the fact, that a constant $\phi$ does not contribute to $A_{\mu}$.\\
%TODO: Some words on non-local redefinitions
Four our computation in chapters \ref{chap:EHT} and \ref{chap:Matter}, we were using the background field method, where we assume a linear split of the \textit{full} metric $g_{\mu\nu}$ into a background metric $\bar{g}_{\mu\nu}$ and a fluctuation field $h_{\mu\nu}$. There is an analogous way of decomposing the fluctuation field in the background field formalism. First, we split $h_{\mu\nu}$ into
\begin{align}
	h_{\mu\nu} = h_{\mu\nu}^{\mathrm{T}} + \frac{1}{d}\ \bar{g}_{\mu\nu}h,
\end{align} 
where $h_{\mu\nu}^{\mathrm{T}}$ is traceless, i.\,e. $\bar{g}^{\mu\nu}h_{\mu\nu}^{\mathrm{T}}=0$ and $h=\bar{g}^{\mu\nu}h_{\mu\nu}$. The traceless part can be further decomposed in flat space using the irreducible representations of the Lorentz group with spins 0, 1 and 2 respectively, but in our case a more sophisticated approach, the so-called \textit{York decomposition} is chosen:
\begin{align}
	h_{\mu\nu} = h_{\mu\nu}^{\text{TT}} + 2\bar{\nabla}_{(\mu}\xi_{\nu)} + \left(\bar{\nabla}_{\mu}\bar{\nabla}_{\nu} - \frac{1}{d} \ \bar{g}_{\mu\nu}\bar{\nabla}^2\right)\sigma + \frac{1}{d} \ \bar{g}_{\mu\nu}h.
\end{align}
Here, $ h_{\mu\nu}^{\text{TT}}$ is a transverse-traceless, spin-2 degree of freedom, $\xi_{\mu}$ is transverse and carries a spin-1 d.\,o.\,f. and $\sigma$ and $h$ have spin-0. The brackets around the indices denote symmetrization, i.\,e. $\bar{\nabla}_{(\mu}\xi_{\nu)} = \frac{1}{2}\left(\bar{\nabla}_{\mu}\xi_{\nu} + \bar{\nabla}_{\nu}\xi_{\mu}\right)$. As before, we want to find the Jacobian $J$ for this variable transformation:
\begin{align}
	\left(\dd h_{\mu\nu}\right) \longrightarrow J	\left(\dd h_{\mu\nu}^{\mathrm{TT}}\right) \left(\dd\xi_{\mu}\right)\left(\dd\sigma\right)\left(\dd h\right).
\end{align}
This is again possible after specifying a suitable normalization of the functional measure as
\begin{align}
	\int (\dd h_{\mu\nu}) \exp\left\{-\mathcal{G}(h, h)\right\} = 1,
\end{align}
where $\mathcal{G}$ is an inner product in the space of symmetric two-tensors, defined as
\begin{equation}
\begin{aligned} 
\mathcal{G}(h, h)&= \int_x \sqrt{\bar{g}} \ \left(h_{\mu \nu} h^{\mu \nu}+\frac{a}{2} h^{2}\right) \\[10pt]
&= \int_x \sqrt{\bar{g}} \ \left[h^{\mathrm{TT}}_{\mu \nu} h^{\mathrm{TT}, \mu \nu}+2 \xi_{\mu}\left(-\bar{\nabla}^{2}-\frac{\bar{R}}{d}\right) \xi^{\mu}\right. \\
&+\left.\frac{d-1}{d} \sigma\left(-\bar{\nabla}^{2}\right)\left(-\bar{\nabla}^{2}-\frac{\bar{R}}{d-1}\right) \sigma+\left(\frac{1}{d}+\frac{a}{2}\right) h^{2} \right] 
\end{aligned}
\end{equation}
in the case of an Einstein type background metric\footnote{A metric is of Einstein type, if $R_{\mu\nu}$ is a constant multiple of $g_{\mu\nu}$, i.\,e. $R_{\mu\nu} = \frac{1}{d} \mathcal{R} g_{\mu\nu}$.}. This yields
\begin{align}
	J=\left(\operatorname{det}_{\xi}\left(-\bar{\nabla}^{2}-\frac{R}{d}\right)\right)^{1 / 2}\left(\operatorname{det}_{\sigma}^{\prime}\left(-\bar{\nabla}^{2}\right)\right)^{1 / 2}\left(\operatorname{det}_{\sigma}\left(-\bar{\nabla}^{2}-\frac{R}{d-1}\right)\right)^{1 / 2}.
\end{align}
Note, that the prime has the same meaning and physical interpretation as in the previous case: If $\sigma$ is constant, it does not contribute to $h_{\mu\nu}$. \\
 For both cases, the decomposition of the general gauge field and the York decomposition of the fluctuation field, appropriate rescalings of the fields $\phi$, $\xi_{\mu}$ and $\sigma$ respectively, help us to cancel the non-trivial Jacobians and to achieve, that all modes have the same mass dimension. For the sake of completeness, we present the rescaled versions of the fields:
 \begin{align}
\hat{\phi} &= \sqrt{-\nabla^2}\ \phi \\[10pt]
 \hat{\xi}_{\mu} &= \sqrt{-\bar{\nabla}^{2}-\frac{\bar{R}}{d}}\  \xi_{\mu} \\[10pt]
  \hat{\sigma} &= \sqrt{-\bar{\nabla}^{2}} \sqrt{-\bar{\nabla}^{2}-\frac{\bar{R}}{d-1}}\ \sigma. 
 \end{align}
 The resulting graviton two-point function, after decomposition of the fluctuation field has the following structure:
\begin{equation} \Gamma^{(2)}_{hh} = 
\begin{pmatrix}
\Gamma^{(2)}_{h^{\mathrm{TT}}h^{\mathrm{TT}}} & 0 & 0 & 0 \\[10pt]
0 & \Gamma^{(2)}_{\xi\xi}  & 0 & 0 \\[10pt]
0 & 0 & \Gamma^{(2)}_{h^{\mathrm{Tr}}h^{\mathrm{Tr}}}  & \Gamma^{(2)}_{h^{\mathrm{Tr}}\sigma} \\[10pt]
0 & 0 & \Gamma^{(2)}_{\sigma h^{\mathrm{Tr}}} & \Gamma^{(2)}_{\sigma\sigma} \\

\end{pmatrix}
\end{equation}
 This concludes our discussion of the York decomposition, as a useful tool to simplify calculations in the background field method.
 \newpage
 \section{Heat-Kernel Techniques}\label{sec:heat-kernel}
We use heat-kernel techniques to evaluate the r.\,h.\,s. of the flow equation (\ref{eqn:Wetterich}), where we need to compute the functional trace over functions depending on the Laplacian on a curved background. In general, the method can be understood as a curvature expansion about a flat background. \\
The general formula to compute such traces is given by
\begin{align}
	\operatorname{Tr} f(\Delta)= N \  \int\kern-1.3em\sum_{\ell} \rho(\ell) f(\lambda(\ell)),
	\label{eqn:heat-kernel}
\end{align}
with some normalization $N$, the spectral values $\lambda(\ell)$ and their corresponding multiplicities $\rho(\ell)$. \\
On flat backgrounds, the computation of (\ref{eqn:heat-kernel}) is simply a standard momentum integral, whereas on curved backgrounds, consider for example a four-sphere $\mathbb{S}^4$ with constant background curvature $r = \frac{\bar{\mathcal{R}}}{k^2} > 0$, the spectrum of the Laplacian is discrete and we need to sum over all spectral values. \\
For our example of $\mathbb{S}^4$, we have
\begin{align}
	\lambda(\ell) = \frac{\ell(3+\ell)}{12}r \qquad \text{and} \qquad \rho(\ell) = \frac{(2\ell + 3)(l+2)!}{6\ell!}.
\end{align}
The normalization is then given by the inverse of the four-sphere-volume $ \left(V_{\mathbb{S}^4}\right)^{-1} = \frac{k^4r^2}{384\pi^2}$. This leads us to the formula for our computation of the r.h.s. of the flow equation on a background with constant positive curvature:
\begin{align}
\operatorname{Tr} f(\Delta)=\frac{k^{4} r^{2}}{384 \pi^{2}} \sum_{\ell=0}^{\infty} \frac{(2 \ell+3)(\ell+2) !}{6 \ell !} f\left(\frac{\ell(3+\ell)}{12} r\right).
\end{align}
This is called spectral sum. For large curvatures $r$ the convergence of the series is rather fast, whereas in the limit $r\rightarrow 0$ one finds exponentially slow convergence.\\
The master equation for heat kernel computations reads
\begin{align}
	\operatorname{Tr} f(\Delta)=\frac{1}{(4 \pi)^{\frac{d}{2}}}\left[\mathbf{B}_{0}(\Delta) Q_{2}[f(\Delta)]+\mathbf{B}_{2}(\Delta) Q_{1}[f(\Delta)]\right]+\mathcal{O}\left(\mathcal{R}^{2}\right),
\label{eqn:master-eqn}
\end{align}
with the heat-kernel coefficients 
\begin{align}
	\mathbf{B}_{n}(\bar{\Delta})=\int_x  \sqrt{g} \  \operatorname{Tr} \mathbf{b}_{n}(\bar{\Delta})
\end{align}
and 
\begin{align}
	Q_{n}[f(x)]=\frac{1}{\Gamma(n)} \int \dd x \ x^{n-1} f(x).\label{eqn:Qfunc}
\end{align}
For computations on $\mathbb{S}^4$, the values for the heat kernel coefficients $\mathbf{B}_n(\bar{\Delta})$ are presented in the following.
\begin{table}[H]
	\centering
	\setlength{\tabcolsep}{5mm}
	\setlength\extrarowheight{2mm}
	\begin{tabular}{c | c c c}
	   & TT & TV & S\\ \hline
	   $\operatorname{Tr} \mathbf{b}_{0}$ & 5 &  3 & 1\\
	  $\operatorname{Tr} \mathbf{b}_{2}$ & $-\frac{5}{6}\mathcal{R}$ & $\frac{1}{4}\mathcal{R}$& $\frac{1}{6}\mathcal{R}$\\
	\end{tabular}
	\caption{Heat-kernel coefficients for transverse-traceless tensors (TT), transverse vectors (TV) and scalars (S) for computations on $\mathbb{S}^4$.}
\end{table}

The basic idea of the proof of equation (\ref{eqn:heat-kernel}) is based on the Laplace transform
\begin{align}
	f(\Delta) = \int_0^{\infty} \dd s \ \operatorname{e}^{-s\Delta}\tilde{f}(s).
\end{align}
We insert this definition of the Laplace transform into equation (\ref{eqn:heat-kernel}) and find
\begin{align}
	\operatorname{Tr} f(\Delta)=\int_{0}^{\infty} \dd s \ \tilde{f}(s) \operatorname{Tr} \operatorname{e}^{-s \Delta}.
\label{eqn:hk2}
\end{align}
The trace on the r.\,h.\,s. is explicitly the trace of the heat kernel. We expand this term as follows:
\begin{align}
	\operatorname{Tr} \operatorname{e}^{-s \Delta}=\frac{1}{(4 \pi)^{\frac{d}{2}}} \sum_{n=0}^{\infty} s^{\frac{n-d}{2}} \mathbf{B}_{n}(\Delta).
\end{align}
This is where the heat-kernel coefficients $\mathbf{B}_n$ become important. We proceed by inserting this expanded version of the heat-kernel trace into equation (\ref{eqn:hk2}) and find:
\begin{equation}
\begin{aligned} 
\operatorname{Tr} f(\Delta) &=\frac{1}{(4 \pi)^{\frac{d}{2}}} \sum_{n=0}^{\infty} \mathbf{B}_{n}(\Delta) \int_{0}^{\infty} \dd s \ s^{\frac{n-d}{2}} \tilde{f}(s) \\[10pt] 
&=\frac{1}{(4 \pi)^{\frac{d}{2}}} \sum_{n=0}^{\infty} \frac{1}{\Gamma\left(\frac{d-k}{2}\right)} \mathbf{B}_{n}(\Delta) \int_{0}^{\infty} \dd t \ t^{\frac{d-n}{2}-1} f(t) \\[10pt]
 &=\frac{1}{(4 \pi)^{\frac{d}{2}}} \sum_{n=0}^{\infty} \mathbf{B}_{n}(\Delta) Q_{\frac{d-n}{2}}[f(t)].
\end{aligned}
\end{equation}
This completes the derivation of the master equation (\ref{eqn:master-eqn}) for heat-kernel computations. Note, that we used the definition of the $Q$-functionals, given in equation (\ref{eqn:Qfunc}) and the relation $\int_{s} s^{-x} \tilde{f}(x)=\frac{1}{\Gamma(x)} \int_{z} z^{x-1} f(z)$.


When investigating matter fields, such as in chapter \ref{chap:Matter}, we often encounter kinetic operators of the form $\tilde{\Delta} = -\nabla^2\cdot\mathbbm{1} +  \mathbf{E}$, where $\mathbf{E}$ is a linear map acting on the spacetime and the internal indices of the fields. In this notation, $\mathbbm{1}$ has to be understood as the identity in the respective field space. \\
If $\left[\Delta, \mathbf{E}\right] = 0$\footnote{In the case of $\left[\Delta, \mathbf{E}\right] \neq 0$, there would be additional terms including (higher order) commutators of $\Delta$ and $\mathbf{E}$ due to the Baker-Campbell-Hausdorff formula.}, we can relate the coefficients of the modified Laplacian $\tilde{\Delta}$ and those of the initially considered operator $-\nabla^2$ via
\begin{align}
	\operatorname{Tr} \operatorname{e}^{-s\left(-\nabla^{2}+\mathbf{E}\right)}=\frac{1}{(4 \pi)^{\frac{d}{2}}} \sum_{k, l=0}^{\infty} \frac{(-1)^{\ell}}{\ell !} \int_x \sqrt{g} \ \operatorname{Tr} \mathbf{b}_{k}(\Delta) \mathbf{E}^{\ell} s^{k+\ell-2}.
\end{align}
This results in the following, modified values for the coefficients we are interested in:
\begin{equation}
\begin{aligned}
	\mathbf{b}_0 &= \mathbbm{1} \\[10pt]
	\mathbf{b}_2 &= \frac{\mathcal{R}}{6}\cdot\mathbbm{1} - \mathbf{E}.
	\label{eqn:coefficients}
\end{aligned}	
\end{equation}
For further study and a more general treatment of the modified Laplacians, including higher order coefficients,\ \cite{CodelloPercacciRahmede2008, Percacci2017} are recommended.
\chapter{Functional methods in Quantum Field Theory}
This chapter introduces the the treatment of Quantum Field Theory (QFT) using functional methods. The main goal is to derive the flow equation for the effective action functional, the generating functional for the one-particle irreducible (1PI) correlation functions. The flow equation was first derived by Christof Wetterich in 1993 \cite{Wetterich1992}. It is the foundation for our treatment of Quantum Gravity in the Asymptotic Safety approach, discussed in more detail later on.
\section{Generating Functionals and Correlation Functions}
We consider a theory setting of $N$ scalar fields $\varphi_a(x), a \in \{1,\dots,N\}$ in $d$-dimensional Euclidean space. The corresponding partition sum in presence of sources $J_a(x)$ reads
\begin{align}
	Z[J] =  \int \D\varphi \operatorname{e}^{-\S + J\cdot\varphi}.
\end{align}
The information content of the partition sum results mainly from the classical action functional $\S$, which determines the classical field equations
\begin{align}
	\frac{\delta \mathcal{S}}{\delta\varphi(x)} = 0.
\end{align}
\textbf{Notation:} The scalar product sums over field components and integrates over all space \dots
\begin{align}
	J\cdot\varphi = \int_x J_a(x) \ \varphi_a(x) = \int_p \tilde{J}_a(p) \ \tilde{\varphi}_a(p)
\end{align}
with
\begin{align}
\int_x = \int_{\mathbb{R}^d} \dd^d x \qquad \text{and} \qquad \int_p = \int_{\mathbb{R}^d} \frac{\dd^d p}{(2\pi)^d}	
\end{align}

Mean field description:
\begin{align}
	\phi := \cf{\varphi} = \eval{\frac{1}{Z}\frac{\delta Z}{\delta J}}_{J=0} = \int \D\varphi \ \varphi \ \operatorname{e}^{-\S + J\cdot\varphi}  
\end{align}

Higher correlations:
\begin{align}
\cf{\varphi_1 \cdots \varphi_n} := \cf{\varphi^n} = \frac{1}{Z}\frac{\delta^n Z}{\delta^n J} = \int \D\varphi \ \overbrace{\varphi_1 \cdots \varphi_n}^{:= \ \varphi^n} \ \operatorname{e}^{-\S + J\cdot\varphi}  
\end{align}

The Schwinger functional $W$ is then defined as
\begin{align}
	Z[J] = \operatorname{e}^{W[J]}
\end{align}

For the special case of $n=2$ the correlation function yields the connected 2-point function which is also known as the propagator $G_{ab}(x,y) = G_{\alpha\beta}$ correlating the field $\varphi_a$ at spacetime point $x$ with the field $\varphi_b$ at $y$.
\begin{align}
	G_{\alpha\beta} &= \frac{\delta^2W[J]}{\delta J_{\alpha}\delta J_{\beta}} = \frac{\delta}{\delta J_{\alpha}}\left(\frac{1}{Z}\frac{\delta Z}{\delta J_{\beta}}\right) \nonumber \\
				&= \frac{1}{Z}\left(\frac{\delta^2Z}{\delta J_{\alpha}\delta J_{\beta}}\right) - \frac{1}{Z^2}\left(\frac{\delta Z}{\delta J_{\alpha}}\right)\left(\frac{\delta Z}{\delta J_{\beta}}\right)\\
				&= \cf{\varphi_{\alpha}\varphi_{\beta}} - \phi_{\alpha}\phi_{\beta} = \cf{\varphi_{\alpha}\varphi_{\beta}}_{\text{c}}	\nonumber	
\end{align}

The Effective Action:\\

The effective action can be obtained by performing a Legendre transform of the Schwinger funtional, i.\,e.:
\begin{align}
\Gamma[\phi] = \underset{J}{\operatorname{sup}}\left\{ \int\limits_x J(x)\phi(x) - \W \right\} = \int\limits_x J_{\text{sub}}(x)\phi(x) - \mathcal{W}[J_{\text{sub}}]
\end{align}

Quantum equation of motion:
\begin{align}
\frac{\delta\Gamma[\phi]}{\delta\phi(x)} = J(x)	
\end{align}

Dyson-Schwinger equation:
\begin{align}
\frac{\delta\Gamma[\phi]}{\delta\phi(x)} = \frac{\delta\mathcal{S}}{\delta\varphi(x)} \left[\varphi = G \cdot \frac{\delta}{\delta\phi} + \phi \right]
\end{align}

\section{The Functional Renormalization Group}
\begin{itemize}
	\item Kadanoff Block-Spin model 
	\item maybe visualization of Ising model + phase transitions
\end{itemize}

\begin{figure}[H]
\centering
\includegraphics{figs/TikZ/block_spin_model}
\caption[Visualization of the Kadanoff Block-Spin model.]{Visualization of the Kadanoff Block-Spin model.\footnotemark}
\label{fig:kadanoff}
\end{figure}
\footnotetext{This visualization is inspired by an image provided in the \href{https://arxiv.org/pdf/cond-mat/0207340.pdf}{PhD thesis} of J.R. Laguna.}
\blindtext

\section{Renormalization Group Consistency}
This section is mainly based on \cite{BraunLeonhardtPawlowski2018}.

Cutoff independence of the full quantum effective action:
\begin{align}
	\Lambda\frac{\dd\Gamma}{\dd\Lambda} = 0
\end{align}

Full effective action in a generic representation:
\begin{align}
	\Gamma[\phi] = \D_{\Lambda}[\phi] + \Gamma_{\Lambda}[\phi]
\end{align}

Formal discussion:
\begin{align}
\Gammak[\phi] = \Gamma_{\Lambda}[\phi] + \int\limits_{\Lambda}^k \frac{\dd k'}{k'} \mathcal{F}_{k'}[\phi]
\end{align}
 
 
\section{Flow Equations for Generating Functionals}
We introduce the RG time scale $t$:
\begin{align}
	\partial_t = \frac{\partial}{\partial\ln(k/\Lambda)} = \frac{k}{\Lambda}\frac{\partial}{\partial(k/\Lambda)} = k \partial_k
\end{align}

\blindtext

\begin{align}
	\partial_t\Gammak[\phi] &= \frac{1}{2}\tr{\frac{1}{\Gammak^{(2)}[\phi] + R_k} \partial_t R_k} \nonumber \\ \phantom{.}  \\
							&= \frac{1}{2}\int\limits_p \frac{1}{\Gammak^{(2)}[\phi] + R_k}(p, -p) \ \partial_t R_k(p^2) \nonumber
\label{eqn:Wetterich}							
\end{align}
This translates directly into the following diagrammic representation:

\begin{figure}[H]
\centering
\begin{gather}
\begin{aligned}
\includegraphics{figs/TikZ/wetterich_equation} 
\end{aligned}
\end{gather}
\end{figure}
where $\otimes = \partial_t R_k$ represents the insertion of the respective regulator. \\


\blindtext

\begin{figure}[H]
\centering
\includegraphics{figs/TikZ/regulator_dependence}
\caption{Flow of $\Gamma_k$ through infinite-dimensional theory space for different regulators.}	
\end{figure}
\chapter{Introduction}
Einsteins theory of General Relativity describes 


\begin{itemize}
	\item Current understanding of Gravity
	\item Need of NP approach due to failure of perturbative quantization
	\item Some words on Wilson, Weinberg etc.
	\item Different approaches to Quantum Gravity (Strings, Loop QG, Causal Dynamical Triangulations etc.)
	\item THIS WORK: Asymptotic Safety, Matter systems etc.
\end{itemize}

The structure of this work is the following. In chapter \ref{chap:QFT} the field theoretical language and the Functional Renormalization Group (FRG) are introduced. A derivation of Wetterich's exact renormalization group equation, a.\,k.\,a. the flow equation,  completes our discussion of non-perturbative approaches to quantum field theory. Chapter \ref{chap:GR} provides the background knowledge on gravity and curved spacetimes. In chapter \ref{chap:EHT}, as a first step towards quantum gravity, the Asymptotic Safety approach is motivated and the flow equation is solved within the Einstein-Hilbert truncation in a transverse-traceless spin-2 graviton approximation. Our calculation is extended in chapter \ref{chap:Matter}, by taking minimally coupled scalar, fermion and gauge fields into account. In chapter \ref{chap:BGindependence} we critically review the background field approximation.
The results are summarized and discussed in chapter \ref{chap:Conclusion}. To conclude this work, an outlook on current progress and open questions in Asymptotic Safety research is presented.\\
 Throughout this thesis we use natural units such that $\hbar = c  \equiv 1$. Einsteins sum convention is implicitly understood: Whenever an index appears twice in a single term, summation of that term over the whole index range is implied unless stated otherwise. As usual, greek indices refer to some $d$-dim. spacetime coordinates, ranging from $0$ to $d-1$, i.\,e. $x^{\mu} = (x^0, x^1, \cdots, x^{d-1})$. For most parts of this thesis, we work in $d=4$ spacetime dimensions.


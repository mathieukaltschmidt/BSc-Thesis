\chapter{Introduction}
Einsteins theory of General Relativity successfully describes gravitational phenomena ranging from quotidian physics to the dynamics of whole galaxies very precisely in terms of the geometry of spacetime. This high precision has been proved once again recently in 2016, with the first observation of gravitational waves by the LIGO and VIRGO collaborations \cite{LIGO2016}. Nevertheless, General Relativity is a \textit{classical} field theory and it is assumed, that the theory breaks down at some characteristic energy scale, i.\,e. the Planck scale
\begin{equation*}
	\Lambda_{\text{Planck}} \approx 10^{19} \ \text{GeV}. 
\end{equation*}
Since decades, physicists work on finding a microscopic,  \textit{quantum} theory of gravity, comparable to the successful description of the other three fundamental forces, namely the electromagnetic and the weak and strong nuclear forces, all unified in the Standard Model of Particle Physics. Such a quantum theory of gravity could provide interesting insides into the physics of e.\,g. the early universe or black holes. Among the most popular proposals for such a theory are e.\,g. String Theory or Loop Quantum Gravity. \\
 One of the main problems in verifying predictions from such theories is, that currently it is all but impossible to probe (quantum) gravitational effects at energies near $\Lambda_{\text{Planck}}$. Particle colliders such as the Large Hadron Collider (LHC) nowadays reach maximum center-of-mass energies of about $\sqrt{s} \approx 14 \ \text{TeV} = 14\cdot 10^3 \ \text{GeV}$. In addition to this experimental problem, it is well known, that the quantization of General Relativity leads to a (perturbatively) non-renormalizable theory  due to the negative mass dimension of Newtons constant
\begin{equation*}
	[G] = -2  
\end{equation*}
in $d=4$ spacetime dimensions \cite{GoroffSanotti1985, tHooftVeltmann1974}. During the last decades, the mathematical toolkit for theoretical physicist has evolved quite rapidly. Especially the development of the Functional Renormalization Group, in its modern formulation introduced by Kenneth Wilson in 1971 \cite{Wilson1971}, offers a powerful, non-perturbative tool to solve path integrals in quantum field theory. \\
Proposed by Steven Weinberg in 1978 \cite{Weinberg1980}, the Asymptotic Safety scenario for Quantum Gravity provides a mechanism for constructing a fundamental quantum field theory of gravity in the language of the Functional Renormalization Group. It aims at generalizing the concept of Asymptotic Freedom, well-known e.\,g. in the context of Yang-Mills theories. The basic idea is, that the ultraviolet (= high energy) behavior of gravity may be governed by a Non-Gaussian Fixed Point (NGFP) of the underlying renormalization group flow. A first successful study of quantum gravity within the Asymptotic Safety scenario was conducted by Martin Reuter in 1996 \cite{Reuter1996}. He derived the flow equations and proved the existence of such a NGFP in a pure-gravity setting in a truncated subsystem. The so-called Einstein-Hilbert truncation he investigated back then, will also be the truncation of our choice for this work. \\
This thesis aims at investigating quantum gravity within the Asymptotic Safety scenario in the Einstein-Hilbert truncation after introducing the concepts needed for a general understanding of the subject. To get a first inside into the underlaying structures, the theory is solved in a pure-gravity setting within a transverse-traceless spin-two graviton approximation. In order to probe a more realistic situation, the gravity-matter sector of the theory has to be taken into account. We include minimally coupled matter fields i.\,e. scalar, fermionic and gauge fields and study their impact on the underlaying fixed point structure of the Einstein-Hilbert truncation. Earlier studies put rather strict constraints on the matter content compatible with Asymptotic Safety, see e.\,g. \cite{DonaEichhornPercacci2013}. In More recently published  research papers, e.\,g. in \cite{MeibohmPawlowskiReichert2015, ChristiansenLitimPawlowskiReichert2018}, strictly less severe constraints have been found. The latter results are based on calculations involving a vertex expansion of the effective average action. The flow of the couplings is then obtained from the flow of the $n$-point functions.
 Throughout this work, all computations will be performed in a background field approximation. Since this approximation has to be treated with care, we have to critically review our computations at the end of the thesis. \\
The structure of this work is the following. In chapter \ref{chap:QFT} the field theoretical language and the Functional Renormalization Group (FRG) are introduced. A derivation of Wetterich's exact renormalization group equation, a.\,k.\,a. the flow equation,  completes our discussion of non-perturbative approaches to quantum field theory. Chapter \ref{chap:GR} provides the background knowledge on gravity and curved spacetimes. In chapter \ref{chap:EHT}, as a first step towards quantum gravity, the Asymptotic Safety approach is motivated and the flow equation is solved within the Einstein-Hilbert truncation in a transverse-traceless spin-two graviton approximation. The inclusion of matter is studied in chapter \ref{chap:Matter}. In chapter \ref{chap:BGindependence} we critically review the background field approximation.
To conclude this work, the results are summarized and discussed in chapter \ref{chap:Conclusion}. \\
 Throughout this thesis we use natural units such that $\hbar = c  \equiv 1$. Einsteins sum convention is implicitly understood: Whenever an index appears twice in a single term, summation of that term over the whole index range is implied unless stated otherwise. As usual, greek indices refer to some $d$-dim. spacetime coordinates, ranging from $0$ to $d-1$, i.\,e. $x^{\mu} = (x^0, x^1, \cdots, x^{d-1})$. For most parts of this thesis, we work in $d=4$ spacetime dimensions.


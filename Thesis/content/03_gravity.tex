\chapter{Curved Spacetimes}
This section is based on \cite{CarrollGR, Schutz2009}.
\section{An Introduction to Spacetime Geometry}


%Some words in differential geometry:  differentiable manifolds, coordinate maps, tangent space, covariant/contravariant vectors + trafos, metric tensor

inner product: 
\begin{align}
	g(X,Y) = g_{\mu\nu}X^{\mu}Y^{\nu} = X^{\mu}Y_{\mu}= g^{\mu\nu}X_{\mu}Y_{\nu} = X_{\mu}Y^{\mu}
\end{align}
we can use the metric tensor to raise and lower spacetime indices, \\
Christoffel symbols (connection):
\begin{align}
\Gamma_{\phantom{\alpha}\mu \nu}^{\alpha}=\frac{1}{2} g^{\mu \lambda}\left(\partial_{\mu} g_{\nu \lambda}+\partial_{\nu} g_{\mu \lambda}-\partial_{\lambda} g_{\mu \nu}\right)	
\end{align}
Geodesic equation:
\begin{align}
	\int d s=\int d \tau \sqrt{g_{\mu \nu} \frac{d x^{\mu}}{d \tau} \frac{d x^{\mu}}{d \tau}}
\end{align}

\begin{align}
	\ddot{x}^{\mu}+\Gamma_{\phantom{\mu}\sigma \rho}^{\mu} \dot{x}^{\sigma} \dot{x}^{\rho}=0
\end{align}
Riemann/Curvature tensor:
\begin{align}
	R_{\phantom{\alpha}\beta \gamma \delta}^{\alpha}=\partial_{\gamma} \Gamma_{\phantom{\alpha}\beta \delta}^{\alpha}-\partial_{\delta} \Gamma_{\phantom{\alpha}\beta \gamma}^{\alpha}+\Gamma_{\phantom{\alpha}\beta \delta}^{\epsilon} \Gamma_{\phantom{\alpha}\epsilon \gamma}^{\alpha}-\Gamma_{\phantom{\alpha}\beta \gamma}^{\epsilon} \Gamma_{\phantom{\alpha}\epsilon \delta}^{\alpha}
\end{align}
Definition using the commutator of covariant derivatives
\begin{align}
	\left[\nabla_{\mu}, \nabla_{\nu}\right] A^{\sigma}=R_{\phantom{\alpha}\rho \mu \nu}^{\sigma} A^{\rho}
\end{align}

Contractions of the Curvature tensor:
\begin{align}
	R_{\mu\nu} = R^{\alpha}_{\phantom{\alpha}\mu\alpha\nu} = g^{\alpha\beta} R^{\beta}_{\phantom{\alpha}\mu\alpha\nu}
\end{align}
Curvature Scalar:
\begin{align}
\mathcal{R} = g_{\mu\nu}R^{\mu\nu} = R^{\mu}_{\phantom{\mu}\mu}
\end{align}



\blindtext


\section{From Geometry to the Einstein Equations}
\Blindtext
The Einstein-Hilbert action:
\begin{align}
	\mathcal{S}_{\text{EH}}[g_{\mu\nu}] = \frac{1}{16\pi G} \int\limits_x \sqrt{-\operatorname{det}g_{\mu\nu}} (\Ricci - 2\Lambda)
\end{align}
Varying this action as usual yields the Einstein equations in absence of matter:
\begin{align}
	G_{\mu\nu} + \Lambda g_{\mu\nu} = 0
\end{align}
where we used $G_{\mu\nu} = \mathcal{R}_{\mu\nu} - \frac{1}{2}g_{\mu\nu}\mathcal{R}$. \\

Diffeomorphism invariance, Lie derivatives:
\begin{align}
	\mathcal{L}_{\omega}\phi = \omega^{\mu}\partial^{\mu}\phi = \omega^{\mu}\nabla^{\mu}\phi
\end{align}

\section{Gravity with Matter} 

Energy-Momentum Tensor:
\begin{align}
	T_{\mu\nu} = \frac{-2}{\sqrt{-\operatorname{det}g_{\mu\nu}}} \frac{\delta\mathcal{S}_{\text{matter}}}{\delta g^{\mu\nu}}
\end{align}

Matter part of the action for a minimally coupled scalar field $\phi$:
\begin{align}
	\mathcal{S}_{\text{matter}}[g_{\mu\nu}, \phi] = -\frac{1}{2} \int\limits_x \sqrt{-\operatorname{det}g_{\mu\nu}}\left( g^{\mu\nu}\nabla_{\mu}\phi\nabla_{\nu}\phi - g_{\mu\nu} V(\phi) \right)
\end{align}

From this, we get the Einstein equations including matter by demanding the variation $\sqrt{-\operatorname{det}g_{\mu\nu}}\frac{\delta\mathcal{S}}{\delta g^{\mu\nu}}$ to vanish. This yields:

\begin{align}
\frac{1}{8\pi G}\left[\Ricci_{\mu\nu} - \frac{1}{2}(\Ricci - 2\Lambda)g_{\mu\nu}\right] = T_{\mu\nu}	
\end{align}


\section{Perturbative Non-Renormalizability of Gravity}

\blindtext

\chapter{Conclusion}\label{chap:Conclusion}
In this thesis, we investigated the Asymptotic Safety scenario for quantum gravity and studied its phase diagram within the Einstein-Hilbert truncation in a background field approximation. We used non-perturbative Functional Renormalization Group techniques to compute the running of Newton's constant $g_k$ and of the cosmological constant $\lambda_k$.  \\

In chapters 1 to 3 we introduced the background knowledge, needed for a general understanding of the conducted calculations. After briefly discussing the general idea of the Asymptotic Safety conjecture, in chapter \ref{chap:EHT} we solved the flow equation for a pure-gravity system in a transverse-traceless spin-two graviton approximation to get a first insight into the underlying structures of the theory setting. The mathematical tools we used to solve the flow equation, such as the York decomposition of the fluctuation field and the heat-kernel techniques to compute the functional traces, were introduced very detailed. We computed the $\beta$-functions for $g_k$ and $\lambda_k$ and determined the fixed points of the flow. Besides the non-interacting Gaussian fixed point at $(g_k, \lambda_k)=(0, 0)$ we found an UV-attractive fixed point at $(g_k, \lambda_k)=(0.86, 0.18)$, providing further evidence for the Asymptotic Safety scenario as a promising candidate for a non-perturbative renormalizable quantum field theory of gravity. In chapter \ref{chap:Matter} we extended our truncation: First of all, the previously neglected trace mode and the Faddeev-Popov ghosts associated with the graviton sector were included to complete the calculation from chapter \ref{chap:EHT}. Then we investigated the impact of minimally coupled scalars, fermions and gauge fields on the Non-Gaussian fixed point. We explained in full detail how to solve the functional traces for all three matter types separately and presented the most important and insightful steps. After finishing the computation of all contributions, we were able to determine the $\beta$-functions for $g_k$ and $\lambda_k$ as a function of the number of matter fields. Neglecting all the contributions from the different anomalous dimensions and expanding the $\beta$-functions in some neighborhood of the Gaussian fixed point up to second order in the couplings, we qualitatively analyzed the behavior of the values for both couplings. We found out, that for an increasing amount of scalar fields, the values of the couplings tend to increase drastically. The fermionic fields and especially the gauge fields seem to have a stabilizing effect on the system. These results are almost in agreement with the results from earlier investigations, where similar conventions have been chosen, see e.\,g \cite{DonaEichhornPercacci2013}.
At the end, in chapter \ref{chap:BGindependence}, we highlighted some problems associated with the background field approximation, e.\,g. the loss of background independence as a consequence of violating the Nielsen identities. \\

This work may not provide fundamentally new results, but it still demonstrates some of the central concepts and calculations related to the subject. Due to the fact, that we worked in a rather simple truncation and chose a Litim-type cutoff, we we able to solve all the problems in this thesis analytically. Interesting modifications of our setup could include the employment of more sophisticated regulators or the inclusion of higher-order curvature terms. Compared to more recent results, the outcomes of our calculations should be treated with care. Nevertheless, this thesis provides a suitable framework for further investigations of asymptotically safe quantum gravity. Based on our discussion in chapter \ref{chap:BGindependence}, one may consider abandoning the background field approximation in future projects. 
{\hypersetup{allcolors=black}
\thispagestyle{plain}

\makeatletter

\begin{center}
\textbf{\large\@title} \\
\vspace{.1cm}
\@author \\
\end{center}

\makeatother

%Abstract in english language
\statement{\large Abstract}
In this work, we investigate asymptotically  safe quantum gravity using non-perturbative Functional Renormalization Group methods. We compute the running of the Newton coupling and the cosmological constant and study the phase diagram of quantum gravity within the Einstein-Hilbert truncation in a background field approximation. In the pure-gravity sector, we find an UV-attractive interacting fixed point, providing further evidence for the Asymptotic Safety scenario for quantum gravity. Additionally, we include minimally coupled scalar, fermionic and gauge fields in our theory setting and analyze their impact on the fixed point structure. We find \dots %TODO: Results...
To conclude this work, we discuss some of the major problems of the background field approximation.


\vfill

%Abstract in german language
\begin{otherlanguage}{german}
\statement{\large Zusammenfassung}
In dieser Arbeit untersuchen wir asymptotisch sichere Quantengravitation mit Methoden der Funktionalen Renormierungsgruppe. Wir berechnen das Laufen der Newton-Konstante und der kosmologischen Konstante und analysieren das Phasendiagramm von Quantengravitation im Rahmen der Einstein-Hilbert Trunkierung in einer Hintergrundfeld-N\"aherung. Die Existenz eines ultraviolett attraktiven Fixpunktes im reinen Gravitationssektor wird nachgewiesen. Dies liefert bereits ein m\"ogliches Indiz f\"ur die Realisierung einer asymptotisch sicheren Quantentheorie der Gravitation. Des Weiteren wird der Einfluss von minimal gekoppelten Skalar-, Fermion- und Eichfeldern auf die Fixpunktstruktur der Theorie untersucht. Wir finden \dots %TODO: Ergebnisse
Am Ende der Arbeit werden einige der Probleme der Hintergrundfeld-N\"aherung erl\"autert.
\end{otherlanguage}
\vfill
\cleardoublepage}

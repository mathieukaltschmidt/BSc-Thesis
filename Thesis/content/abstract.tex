{\hypersetup{allcolors=black}
\thispagestyle{plain}

\makeatletter

\begin{center}
\textbf{\large\@title} \\
\vspace{.1cm}
\@author \\
\end{center}

\makeatother

%Abstract in english language
\statement{\large Abstract}
\blindtext
%In this work we discuss the Asymptotic Safety approach as a possible realization of a theory of quantum gravity based on path integral quantization using functional renormalization group methods. First, the exact renormalization group equation is solved in a spin-2 graviton approximation using the background field formalism and the respective fixed point structure is analyzed. In the second part, we investigate minimally coupled scalar, fermion and gauge fields and their impact on the system. Finally, we discuss the validity of our computations in the background field method and present the fluctuation field formalism as a modern, alternative approach. 

\vfill

%Abstract in german language
\begin{otherlanguage}{german}
\statement{\large Zusammenfassung}
\blindtext
%In dieser Arbeit wird der Asymptotic-Safety Zugang als m\"oglicher Ansatz zur Realisierung einer Theorie der Quantengravitation im Rahmen der Pfadintegral-Quantisierung mit Methoden der funktionalen Renormierungsgruppe untersucht. Die exakte Renormierungsgruppengleichung wird zun\"achst in einer Spin-2-Graviton N\"aherung im Hintergrundformalismus gel\"ost und die resultierende Fixpunkt-Struktur analysiert. Im zweiten Teil der Arbeit wird der Einfluss von minimal gekoppelten Skalar-, Fermion- und Eichfeldern auf das System \"uberpr\"uft. Abschlie\ss end wird die Hintergrundfeld-Methode kritisch hinterfragt und mit der Fluktuationsfeld-Methode ein moderner, alternativer Zugang pr\"asentiert.
\end{otherlanguage}
\vfill
\cleardoublepage}

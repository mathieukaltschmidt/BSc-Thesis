\chapter{Quantum Gravity in the Einstein-Hilbert Truncation}
%TODO: GAUGE FIXING! Faddeev-Popov
\section{RG approach to Quantum Gravity}
Flow equation for QG:
\begin{align}
	\partial_t\Gammak[\overline{g}, \Phi] = \frac{1}{2} \operatorname{Tr}\ G_{\text{hh}}[\Phi] \ \partial_tR_k - \operatorname{Tr}\ G_{\text{c}\overline{\text{c}}}[\Phi] \ \partial_tR_k
\end{align}

\section{Einstein-Hilbert truncation}
We want to solve the Flow equation (\ref{eqn:QGflow}) approximately. All terms that are invariant under the imposed symmetry, i.\,e. invariant under diffeomorphism transformations need to be taken into account. \\

Easiest truncation takes only the scalar curvature $\mathcal{R}$ and the cosmological constant $\Lambda$ into account (No higher order terms \dots) and was performed by Martin Reuter in 1993 \cite{ReuterSaueressig2002}. \\


This truncation reads
\begin{align}
	\Gammak = 2\kappa^2Z_k \int_x \sqrt{\operatorname{det}g} \ [-\mathcal{R} + 2\Lambda_k] + \mathcal{S}_{\text{gf}} + \mathcal{S}_{\text{gh}}
\label{eqn:QGflow}
\end{align}
with 
\begin{align}
	\kappa^2 = \frac{1}{32\pi G}, \qquad\qquad G_k = GZ^{-1}_k
\end{align}

Linear gauge fixing $F_{\mu}$ and corresponding ghost term induced by the Faddeev-Popov trick:
\begin{align}
\mathcal{S}_{\text{gf}} &= \frac{1}{2\alpha} \int_x \sqrt{\operatorname{det}\overline{g}} \  \overline{g}^{\mu\nu} F_{\mu}F_{\nu}	\nonumber\\
\phantom{.} \\
\mathcal{S}_{\text{gh}} &= \int_x \sqrt{\operatorname{det}\overline{g}} \  \overline{g}^{\mu\mu'} \overline{g}^{\nu\nu'}\overline{c}_{\mu'} \mathcal{M}_{\mu\nu}  c_{\nu'} \nonumber
\end{align}
with the Faddeev-Popov operator $\mathcal{M}_{\mu\nu}(\overline{g},h)$ for the gauge fixing $F_{\mu}(\overline{g},h)$. \\
A linear, de-Donder type gauge fixing with 
\begin{align}
	F_{\mu} &= \overline{\nabla}^{\nu}h_{\mu\nu} - \frac{1+\beta}{4}\overline{\nabla}_{\mu}h^{\nu}_{\phantom{\nu},\nu} \nonumber \\
	\phantom{.} \\
	\mathcal{M}_{\mu\nu} &= \overline{\nabla}^{\rho}(g_{\mu\nu}\nabla_{\rho} + g_{\rho\nu}\nabla_{\mu}) - \overline{\nabla}_{\mu}\nabla_{\nu}, \nonumber
\end{align}
is employed, where $\beta=1$ and $\alpha\rightarrow 0$ represents a fixed point of the RG flow. Note, that the limit  $\alpha\rightarrow 0$ is performed after the gauge fixing process.  \\

anomalous dimension: 
\begin{align*}
	\eta_g = -\frac{\partial_t Z_k}{Z_k} = -\partial_t \ln Z_k
\end{align*}

dimensionless renormalized cosmological constant:
\begin{align*}
	\lambda_k = \Lambdak k^{-2}
\end{align*}

dimensionless renormalized cosmological constant:
\begin{align*}
	g_k = G_k k^{d-2} = \frac{Gk^{d-2}}{Z_k}
\end{align*}

corresponding beta function:
\begin{align}
	\beta_g = \partial_t g_k = \left(d-2 + \eta_g\right)g_k
\end{align}

maximally symmetric space:
\begin{align}
	\overline{\mathcal{R}}_{\mu\nu} = \frac{1}{d} \ \overline{g}_{\mu\nu} \overline{\mathcal{R}}
\end{align}
\begin{align}
	\overline{\mathcal{R}}_{\mu\nu\rho\sigma} = \frac{1}{d(d-1)} \ (\overline{g}_{\mu\rho}\overline{g}_{\nu\sigma} - \overline{g}_{\mu\sigma}\overline{g}_{\nu\rho}) \overline{\mathcal{R}}
\end{align}

suitable tensor basis:
\begin{align}
	h_{\mu\nu} = h_{\mu\nu}^{\text{TT}} + \overline{\nabla}_{\mu}\xi_{\nu} + \left(\overline{\nabla}_{\mu}\overline{\nabla}_{\nu} - \frac{1}{d} \ \overline{g}_{\mu\nu}\overline{\Delta}\right)\sigma + \frac{1}{d} \ \overline{g}_{\mu\nu}h
\end{align}
As a first approximation, we only take the contribution from the spin-two graviton mode $h_{\mu\nu}^{\text{TT}}$ into account. This is motivated by the fact, that this mode carries the the most degrees of freedom.\\ %TODO: Extend this explanation (exercise sheet...)
In this setting, we want to solve the Wetterich equation (\ref{eqn:Wetterich}) by computing the left hand side and the right hand side separately and extract the $\beta$-functions for the Newton coupling $g_k$ and the cosmological constant $\lambda_k$ by a comparison of all terms of order $\sim\sqrt{\operatorname{det}g}$ and $\sim \sqrt{\operatorname{det}g} \ \mathcal{R}$. Here, only the most important steps of the calculation are presented. For the complete calculation have a look at the Appendix A. \\%TODO:Referencing the Appendix. 
In our spin-two graviton mode approximation, we don't have to deal with the gauge-fixing and ghost parts ocuring in the effective action. The simplified version of equation (\ref{eqn:QGflow}) reads
\begin{align}
	\Gamma_{k, h^{\text{TT}}} = 2\kappa^2Z_k \int_x \sqrt{\operatorname{det}g} \ [-\mathcal{R} + 2\Lambda_k].
\end{align}
We start by computing the transverse-traceless graviton two-point function
\begin{align}
\Gamma_{h^{\text{TT}}h^{\text{TT}}}^{(2)} = \frac{Z_k}{32\pi}\left(\overline{\Delta} - 2\Lambda_k+\frac{2}{3}\overline{\mathcal{R}}\right).
\end{align}
Using a regulator of form
\begin{align}
R_k  = \eval{\Gamma_{h^{\text{TT}}h^{\text{TT}}}^{(2)}}_{\Lambda_k=\overline{\mathcal{R}}=0} \cdot r_k\left(\frac{\overline{\Delta}}{k^2}\right) = \frac{Z_k}{32\pi}\overline{\Delta}\left(\frac{k^2}{\overline{\Delta}}-1\right)\Theta\left(1-\frac{\overline{\Delta}}{k^2}\right), \nonumber
\end{align}
with a Litim-type cutoff
\begin{align}
r_k(x) = \left(\frac{1}{x}-1\right)\Theta(1-x),
\end{align}
we are directly able to compute the l.h.s. of the Wetterich equation, i.\,e. the scale derivative of the effective average action:
\begin{align}
	\partial_{t}\Gamma_{k,h^{\text{TT}}} = 2\kappa^2 Z_k\int_x \sqrt{\operatorname{det}g} \left\{\eta_g\mathcal{R}+2\left(k^2(\partial_t\lambda_k) + \Lambda_k(2 - \eta_g)\right)\right\}
\end{align}
One finds the $\beta$-function for the Newton coupling without performing the analysis of the Wetterich equation, i.\,e. 
\begin{align}
\beta_g = \partial_t g_k = \partial_t\left(\frac{G\cdot k^2}{Z_k}\right) = g_k\left(2+\eta_g\right).
\end{align}
%TODO: RHS calculation using heat-kernels.
For the cosmological constant, comparing the $\sqrt{\operatorname{det}g}$ terms yields
\begin{align}
	\beta_{\lambda} = \partial_t\lambda_k = -4\lambda_k + \frac{\lambda_k}{g_k} \partial_t g_k + \frac{10}{4\pi}g_k\frac{1-\frac{\eta_g}{6}}{1-2\lambda_k}.
\end{align}
where the anomalous dimension $\eta_g$ is determined by comparing the $\sqrt{\operatorname{det}g}\mathcal{R}$ terms:
\begin{align}
\eta_g = -\frac{5}{3\pi} \left(\frac{1-\frac{\eta_g}{4}}{1-2\lambda_k} + 2\frac{1-\frac{\eta_g}{6}}{(1-2\lambda_k)^2}\right).	
\end{align}

Using \verb|Mathematica| to solve this system of coupled differential equations, we arrive at the following fixed point values for the Newton coupling and the cosmological constant:
\begin{align}
	(g_k^*, \lambda_k^*) = (0.86, 0.18).
\end{align}
The corresponding critical exponents, i.\,e. the eigenvalues of the stability matrix evaluated at the fixed point, are given by
\begin{align}
	\theta_{1,2} = 2.9 \pm 2.6i
\end{align}
%TODO: Plot of the Fixed Point (Vector field??)
%TODO: Interpretation and transition to next chapter (MATTER)
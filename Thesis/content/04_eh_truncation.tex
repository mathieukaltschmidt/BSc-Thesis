\chapter{Functional Renormalization and Quantum Gravity}\label{chap:EHT}
After the formal introduction of the physical and mathematical concepts in the last chapters, we are now able to motivate and formulate the key idea of the Asymptotic Safety approach to quantum gravity, which aims at finding a quantum field theoretical description of gravity within the language of the Functional Renormalization Group. This chapter briefly discusses the requirements for the existence of such a theory. We proceed by solving the flow equation for quantum gravity within the Einstein-Hilbert truncation in a transverse-traceless spin-$2$ graviton approximation and investigate its fixed point structure. 
\section{Asymptotic Safety}
In 1978, Steven Weinberg proposed the \cite{Weinberg1980}
\textit{Asymptotic Safety}, also often referred to as \textit{Non-Perturbative Renormalizability}, generalizes the concept of Asymptotic freedom.The requirements for an asymptotically safe quantum field theory of gravity can be summarized as follows: 
\begin{enumerate}
	\item The existence of an UV-attractive NGFP of the Renormalization Group Flow of $\Gammak$ has to be guaranteed. 
	
	\item For the predictivity of the theory, it is crucial to be able to fix the trajectory $\Gammak$ by a finite amount of measurements, i.\,e. the corresponding UV hypersurface $\Sigma_{\mathrm{UV}}$ of the NGFP should be of finite dimension: 
	\begin{equation*}
		\operatorname{dim}\Sigma_{\mathrm{UV}} < \infty.
	\end{equation*} 
\end{enumerate} 


For a more detailed discussion of current Asymptotic Safety research, we refer the interested reader to \cite{Eichhorn2018}. Recently, very detailed textbooks covering both, the physical and the mathematical concepts of Asymptotic Safety have been released. For a very detailed treatment of the subject see e.\,g. \cite{Percacci2017} or \cite{ReuterSaueressig_2019}.
\section{Einstein-Hilbert Truncation}
Solving the flow equation analytically is nothing but impossible. Therefore it is unavoidable to truncate the initially infinite dimensional theory space to a finite subspace, to be able to find approximated solutions. It is important, that all terms, that are invariant under the imposed symmetry, i.\,e. invariant under diffeomorphism transformations need to be taken into account. The easiest truncation fulfilling this requirement is the \textit{Einstein-Hilbert truncation}. In 1996, Martin Reuter was the first to investigate quantum gravity in the Einstein-Hilbert truncation \cite{Reuter1996}. This truncation takes only the scalar curvature $\Ricci$ and the cosmological constant $\Lambda$ into account\footnote{More recently, truncations including higher-order curvature terms ($\Ricci^2$, $f(\Ricci)$, $R^{\mu\nu}R_{\mu\nu}$ \dots) have been investigated, see e.\,g. \cite{AlkoferSaueressig2018}}. The full Einstein-Hilbert  truncation reads:
\begin{align}
	\Gammak = 2\kappa^2Z_k \int_x \sqrt{g} \ [-\mathcal{R} + 2\Lambda_k] + \mathcal{S}_{\text{gf}} + \mathcal{S}_{\text{gh}}.
\label{eqn:EHtruncation}
\end{align}
with 
\begin{align}
	\kappa^2 = \frac{1}{32\pi G}, \qquad\qquad G_k = GZ^{-1}_k
\end{align}



anomalous dimension: 
\begin{align*}
	\eta_g = -\frac{\partial_t Z_k}{Z_k} = -\partial_t \ln Z_k
\end{align*}

dimensionless renormalized cosmological constant:
\begin{align*}
	\lambda_k = \Lambdak k^{-2}
\end{align*}

dimensionless renormalized cosmological constant:
\begin{align*}
	g_k = G_k k^{d-2} = \frac{Gk^{d-2}}{Z_k}
\end{align*}

corresponding beta function:
\begin{align}
	\beta_g = \partial_t g_k = \left(d-2 + \eta_g\right)g_k
\end{align}

maximally symmetric space:
\begin{align}
	\bar{\mathcal{R}}_{\mu\nu} = \frac{1}{d} \ \bar{g}_{\mu\nu} \bar{\mathcal{R}}
\end{align}
\begin{align}
	\bar{\mathcal{R}}_{\mu\nu\rho\sigma} = \frac{1}{d(d-1)} \ (\bar{g}_{\mu\rho}\bar{g}_{\nu\sigma} - \bar{g}_{\mu\sigma}\bar{g}_{\nu\rho}) \bar{\mathcal{R}}
\end{align}

suitable tensor basis:

%TODO: Adapt this to the definition given in the Appendix 
As a first approximation, we only take the contribution from the spin-two graviton mode $h_{\mu\nu}^{\text{TT}}$ into account. This is motivated by the fact, that this mode carries the the most degrees of freedom.\\ %TODO: Extend this explanation (exercise sheet...)
In this setting, we want to solve the Wetterich equation (\ref{eqn:Wetterich}) by computing the l.\,h.\,s. and the r.\,h.\,s. separately and extract the $\beta$-functions for the Newton coupling $g_k$ and the cosmological constant $\lambda_k$ by a comparison of all terms of order $\sim\sqrt{g}$ and $\sim \sqrt{g} \ \mathcal{R}$. \\
In our spin-two graviton mode approximation, we don't have to deal with the gauge-fixing and ghost parts ocuring in the effective action. The simplified version of equation (\ref{eqn:QGflow}) reads
\begin{align}
	\Gamma_{k, h^{\text{TT}}} = 2\kappa^2Z_k \int_x \sqrt{g} \ [-\mathcal{R} + 2\Lambda_k].
\end{align}
We start by computing the transverse-traceless graviton two-point function
\begin{align}
\Gamma_{h^{\text{TT}}h^{\text{TT}}}^{(2)} = \frac{Z_k}{32\pi}\left(\bar{\Delta} - 2\Lambda_k+\frac{2}{3}\bar{\mathcal{R}}\right).
\end{align}
Using a regulator of the form
\begin{align}
R_k  = \eval{\Gamma_{h^{\text{TT}}h^{\text{TT}}}^{(2)}}_{\Lambda_k=\bar{\mathcal{R}}=0} \cdot r_k\left(\frac{\bar{\Delta}}{k^2}\right) = \frac{Z_k}{32\pi}\bar{\Delta}\left(\frac{k^2}{\bar{\Delta}}-1\right)\theta\left(1-\frac{\bar{\Delta}}{k^2}\right), \nonumber
\end{align}
with a Litim-type cutoff
\begin{align}
r_k(y) = \left(\frac{1}{y}-1\right)\theta(1-y), \label{eqn:Litim}
\end{align}
as discussed in chapter (\ref{chap:QFT}), we are directly able to compute the l.\,h.\,s. of the Wetterich equation, i.\,e. the scale derivative of the effective average action:
\begin{align}
	\partial_{t}\Gamma_{k,h^{\text{TT}}} = 2\kappa^2 Z_k\int_x \sqrt{g} \left\{\eta_g\mathcal{R}+2\left(k^2(\partial_t\lambda_k) + \Lambda_k(2 - \eta_g)\right)\right\}
\end{align}
One can extract the $\beta$-function for the Newton coupling without performing the analysis of the Wetterich equation, i.\,e. 
\begin{align}
\beta_g = \partial_t g_k = \partial_t\left(\frac{G\cdot k^2}{Z_k}\right) = g_k\left(2+\eta_g\right).
\end{align}

The computation of the r.\,h.\,s. of the flow equation is more complicated because it involves the computation of a trace of a function depending on the Laplacian on a curved background. We can use heat-kernel techniques to solve such equations. Heat-kernel computations are based on a curvature expansion in powers of the curvature scaler $\mathcal{R}$. For more details, have a look at the appendix (\ref{sec:heat-kernel}). As a first step, we simplify the trace expression as much as possible.

\begin{align}
\operatorname{Tr}\left[\frac{1}{\Gammak^{(2)} + R_k}\partial_t R_k\right] & = \operatorname{Tr}\left[\frac{\partial_t\left(\frac{Z_k}{32\pi}\bar{\Delta}\right)r_k}{\left(\frac{Z_k}{32\pi}\right)\left(\bar{\Delta} - 2\Lambda_k + \frac{2}{3}\bar{\mathcal{R}}\right) + \left(\frac{Z_k}{32\pi}\bar{\Delta}\right) r_k}\right]	\nonumber \\
\phantom{.} \\
&= \operatorname{Tr}\left[\frac{\bar{\Delta}\left(\partial_t r_k - \eta_g r_k\right)}{\bar{\Delta}(1+r_k)-2\Lambda_k+\frac{2}{3}\bar{\mathcal{R}}}\right] \nonumber
\end{align}

We expand this expression around vanishing curvature and get
\begin{align}
\resizebox{.9 \textwidth}{!}{$
\operatorname{Tr}\left[\frac{1}{\Gammak^{(2)} + R_k}\partial_t R_k\right] = \operatorname{Tr}\left[\frac{\bar{\Delta}\left(\partial_t r_k - \eta_g r_k\right)}{\bar{\Delta}(1+r_k)-2\Lambda_k}\right] - \frac{2}{3}\bar{\mathcal{R}}\operatorname{Tr}\left[\frac{\bar{\Delta}\left(\partial_t r_k - \eta_g r_k\right)}{\left(\bar{\Delta}(1+r_k)-2\Lambda_k\right)^2}\right] + \mathcal{O}(\mathcal{R}^2)$
}
\end{align}
Now we are able to evaluate these two terms separately using heat-kernel techniques. One finds for the first term
\begin{align}
\operatorname{Tr}\left[\frac{\bar{\Delta}\left(\partial_t r_k - \eta_g r_k\right)}{\bar{\Delta}(1+r_k)-2\Lambda_k}\right] = \frac{1}{(4\pi)^2}\int_x \sqrt{g} \left[5\Phi_2^1(-2\Lambda_k) - \frac{5}{6}\bar{\mathcal{R}}\Phi^1_1(-2\Lambda_k)\right],
\end{align}
with the threshold functions 
\begin{equation}
\begin{aligned}
	\Phi_n^p(\omega) &= \frac{1}{\Gamma(n)}\int_0^{\infty}\dd z \ z^{n-1} \frac{z(-2zr_k(z)-\eta_{\Psi}r_k(z))}{(z(1+r_k(z))+\omega)^p}\\[10pt] 
	&= \frac{1}{\Gamma(n)}\frac{1}{\left(1+\omega\right)^p}\left(\frac{2}{n} - \frac{\eta_{\Psi}}{n(n+1)}\right)
\end{aligned}
\label{eqn:threshold}
\end{equation}
In the last step, the evaluated the threshold functions for the Litim-type shape function. We used $\eta_\Psi$, since we want to keep this formula as general as possible and we will use it multiple times for different fields throughout this thesis.\\
Analogously, the second term in our expansion reads
\begin{align}
	-\frac{2}{3}\bar{\mathcal{R}}\operatorname{Tr}\left[\frac{\bar{\Delta}\left(\partial_t r_k - \eta_g r_k\right)}{\left(\bar{\Delta}(1+r_k)-2\Lambda_k\right)^2}\right] = -\frac{10}{3}\frac{\bar{\mathcal{R}}}{(4\pi)^2}\int_x  \sqrt{g} \frac{1-\frac{\eta_g}{6}}{(1-2\lambda_k)^2}.
\end{align}
For the cosmological constant, comparing the $\int\sqrt{g}$ terms yields
\begin{align}
	\beta_{\lambda} = \partial_t\lambda_k = -4\lambda_k + \frac{\lambda_k}{g_k} \partial_t g_k + \frac{5}{4\pi}g_k\frac{1-\frac{\eta_g}{6}}{1-2\lambda_k},
\end{align}
where the anomalous dimension $\eta_g$ is determined by comparing the $\int\sqrt{g}\mathcal{R}$ terms:
\begin{align}
\eta_g = -\frac{5}{3\pi} \left(\frac{1-\frac{\eta_g}{4}}{1-2\lambda_k} + 2\frac{1-\frac{\eta_g}{6}}{(1-2\lambda_k)^2}\right).	
\end{align}

The solution of this system of coupled differential equations is evaluated using \verb|Python3| and \verb|Wolfram Mathematica|. We arrive at the following fixed point values for the Newton coupling and the cosmological constant:
\begin{align}
	(g_k^*, \lambda_k^*) = (0.86, 0.18).
\end{align}
The corresponding critical exponents, i.\,e. minus the eigenvalues of the stability matrix evaluated at the fixed point, are given by the complex conjugated pair
\begin{align}
	\theta_{1,2} = 2.9 \pm 2.6i. 
\end{align}

\begin{figure}[t]
\centering
	\includegraphics[width=\textwidth]{figs/Plots/EH_NoMatter}
	\caption[RG flow diagram for the Einstein-Hilbert truncation in TT approximation]{RG flow diagram  for the Einstein-Hilbert truncation in TT approximation as computed in this work. The flow points towards the infrared.}\end{figure}

%TODO: Interpretation and transition to next chapter (MATTER)